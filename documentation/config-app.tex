\documentclass[12pt]{article}

\usepackage{fullpage}
\usepackage{hyperref}

\title{Resume Generator Configuration App Conceptual Design}
\author{Jon Weatherspoon}
\date{\today}

\begin{document}
    \maketitle 

    \tableofcontents

    \section{High Level Flow}
    Generating a new resume shall be done using the configuration app. 
    If the user does not have a resume shell file (template file) ready, they 
    will have to create one. 
    
    % TODO: import resume shell interface 
    % { "template": string, "theme": string, "regions": list<object> }

    Once they have a template file, the user will load it into the configuration 
    app through the file menu (or through the button in the preview pane). Upon 
    loading the file, the application will grab the template and theme ids from 
    it and attempt to find a matching JSX file / theme configuration respectively. 
    If both are found (or just the JSX file if the theme is an empty string), the 
    file parsing will continue. If there are any regions with sections present, 
    they will be parsed into the "editor" section of the configuration app. 

    The editor section will be presented as a hierarchy following the same structure
    as the opened file. The base of the hierarchy will be the template / configuration's 
    name, followed by a list of named regions. Each region may have zero or more components
    and each component may have zero or more editable properties. 

    The user shall be able to click on a region to add a new component to it. When doing this, 
    a dialog should appear, listing out the available components. These will be pulled from 
    some data provider (likely a json file at first for simplicity, database in the future). 

    The user shall be able to expand a component to show its properties. The user shall be 
    able to edit those properties depending on their data types. The editors displayed in 
    the application will depend on the data type in the property definition. See 
    \ref{sec:building-the-editor} for details.

    \section{Creating a New Resume Configuration File}
    If no resume configuration file has been loaded, the preview / editor sections will be replaced 
    by a canvas with the text "You must load or create a resume configuration file to view content."
    Underneath the text will be two buttons. The first will spawn a dialog to create a new resume 
    configuration file and automatically load it into the application. The second button will spawn a 
    file browse dialog and allow the user to select an existing configuration file.  

    \subsection{Creation Process}
    When creating a new resume configuration file, a dialog will spawn asking for the following information:
    \begin{itemize}
        \item The desired filename (required)
        \item The location to save the file to (required)
        \item The resume template to use (required)
        \item The theme definition to use (optional)
    \end{itemize}

    The user will have to select the resume template from a list retrieved by some provider (likely a JSON 
    mapping at first, then database in the future). 

    Once the information has been entered and the user clicks okay, the file will be created and loaded 
    into the application. The resume's template cannot be changed at this point. If the user wishes to 
    change the template, they will have to create a new resume and select that template.

    \section{Building the Editor Section}
    \label{sec:building-the-editor}
    When loading an existing resume configuration file, the application will have to build the editor / 
    preview sections to match the properties defined in the file. To do this, we will follow these steps:
    \begin{enumerate}
        \item Load the configuration file as a javascript object 
        \item Verify the configuration's template id is valid. Fail if it is unknown
        \item If present, attempt to verify the configuration's theme id is valid. 
        \begin{itemize}
            \item If not, set it to the empty string
        \end{itemize}
        \item Add a node to the editor section containing the configuration name 
        \item Under it, add a ThemeEditor node for the theme 
        \item For each region defined in the configuration:
        \begin{enumerate}
            \item Add a new region node with the corresponding name 
            \item For each component in the region:
            \begin{enumerate}
                \item Look up the component definition from the component id 
                \begin{itemize}
                    \item If not found, fail?
                \end{itemize}
                \item For each property in the component definition:
                \begin{enumerate}
                    \item Add a new property editor node based on the property's data type.
                    \item If the property exists on the component defined in the resume 
                    configuration file, set up the new editor with that value. Otherwise, 
                    use the default value from the component definition.
                \end{enumerate}
            \end{enumerate}
        \end{enumerate}
    \end{enumerate}   
    
    \subsection{ComponentFactory}
    The ComponentFactory will be used to transform a component id to a JSX element that 
    can be added to the virtual DOM. We shall create a IResumeComponent interface
    that will include a component id (string) and an implementation of the factory method.
    (probably more?)

    A ComponentFactory class shall be created and it shall import all existing 
    IResumeComponents upon initialization. When parsing a configuration file, we 
    will query the factory for a component with the given name. If found, the factory 
    will create it and pass it back to us. Our job is to then add it to the preview 
    section of the application. 

    Each new type of component will have to implement the interface and the dev will add 
    an import for the component into the ComponentFactory definition. Or perhaps, the 
    ComponentFactory class will be a singleton or something and each component type will 
    just register to it? 

    \subsection{PropertyEditorFactory}


    \subsection{Editor Section Layout}
    The resume editor section will be a hierarchical control that uses nested MUI lists.
    The root node of the editor will contain the name of the loaded resume template file. 
    Under it will be an expandable "Regions" node and an editable "Theme" property. Expanding 
    the "Regions" node will show all regions defined by the selected template.

    \subsection{Base Editor Implementations}
    \subsubsection{TextEditor}

    \subsubsection{DateEditor}

    \subsubsection{ListEditor<T>}

    \subsection{Specialized Editor Implementations}
    \subsubsection{ThemeEditor}

    \subsection{Saving a Configuration}

\end{document}